\documentclass[12pt,a4paper]{article}

\usepackage[T1]{fontenc}
\usepackage[margin=2cm]{geometry}
\usepackage{float}
\usepackage{graphicx}
\graphicspath{ {img/} }
\usepackage{color}
\usepackage{hyperref}
\hypersetup{
    colorlinks,
    citecolor=black,
    filecolor=black,
    linkcolor=black,
    urlcolor=black
}

\title{Documento di candidatura}
\author{Capitolato C4 - Guida Michelin @ social - Zero12}
\date{7 Aprile 2022}
\renewcommand*\contentsname{Indice}
\begin{document}

\begin{figure}[H]
	\centering
	\includegraphics[width=8cm]{FoxyByte_logo_transparency.png}
\end{figure}

\begin{center}
	\textbf{} foxybyte.swe@gmail.com
\end{center}

\begin{table}[H]
	\centering
	\begin{tabular}{c|c c c}
		\textbf{Componenti} & Biasotto     & Luca  & 1162290 \\
		                    & Bosinceanu & Ecaterina & 1169669 \\
		                    & Ferrari   & Gianluca      & 1174586 \\
		                    & Fincato   & Alessandro   & 1201264 \\
		                    & Hida   & Denisa   & 1204284 \\
		                    & Lauriola   & Pietro   & 1224820 \\
		                    & Uderzo   & Marco   & 1201290 \\
	\end{tabular}
\end{table}


\begin{titlepage}
\maketitle
\end{titlepage}

\tableofcontents

\section{Resoconto dell'incontro col proponente}
	\subsection{Informazioni generali}
		\subsubsection{Luogo e data}
			L'incontro si è svolto il giorno 15/03/2022 dalle 16.30 alle 17.30 in sede virtuale (Google Meet).
		\subsubsection{Partecipanti}
			\begin{itemize}
			    \item Interni
			    	\begin{itemize}
					    \item Biasotto Luca
		                \item Bosinceanu Ecaterina
		                \item Ferrari Gianluca
		                \item Fincato Alessandro
		                \item Hida Denisa
		                \item Lauriola Pietro
		                \item Uderzo Marco
					\end{itemize}
			    \item Esterni
			    	\begin{itemize}
					    \item Michele Massaro
					\end{itemize}
			\end{itemize}
	\subsection{Descrizione}
		L'obiettivo del progetto è la creazione di una piattaforma che permetta di raccogliere recensioni da post e video di Instagram e TikTok relativi ad un luogo indicato. Deve permettere di creare una mappa dei luoghi di interesse ed indicare gli utenti da seguire per creare una guida. 
	\subsection{Linguaggi e strumenti}
		Sarà necessario costruire un database servendosi di un crawler per il recupero dei dati. Nello specifico, serviranno strumenti di machine learning per estrarre i dati utili da video e post. Gli strumenti proposti sono:
		\begin{itemize}
			\item AWS Rekognition: è un servizio Amazon di visione artificiale, capace di riconoscere ed estrarre informazioni da video e immagini;
			\item AWS Comprehend: è un servizio Amazon per l'elaborazione del linguaggio naturale e rilevare informazioni all'interno di un testo.
		\end{itemize}

		I linguaggi consigliati per lo sviluppo in back-end sono:
		\begin{itemize}
			\item Node.js;
			\item Typescript.
		\end{itemize}
		Per quanto riguarda il front-end si potrà scegliere tra Angular e React, dunque è necessaria conoscere:
		\begin{itemize}
			\item JavaScript;
			\item HTML;
			\item CSS.
		\end{itemize}

	\subsection{Criticità}
	I fattori critici che sono sovvenuti al gruppo sono i seguenti:
		\begin{itemize}
			\item Non esistono applicazioni simili da cui trarre spunto;
			\item Poca famigliarità con i crawler e con i servizi AWS;
			\item Nonostante le informazioni vengano estratte da post, foto e vide pubblici c'è il rischio di blocchi e limitazioni da parte dei social; inoltre, per lo stesso motivo, potrebbe non essere possibile implementare tutte le funzionalità richieste.
			\item Dispendioso in termini di tempo in quanto necessita di molto studio degli strumenti ed i linguaggi che il gruppo non conosce;
		\end{itemize}

	\subsection{Considerazioni}
	\begin{itemize}
		\item richiede un'analisi molto precisa ed approfondita su come costruire l'architettura e ragionare sul modo più efficace per visualizzare i dati lato utente;
		\item scegliere se è più conveniente utilizzare un'architettura server o serverless;
	\end{itemize}

\section{Motivazioni della scelta}
Di seguito le motivazioni che ci hanno portato a scegliere questo capitolati:
	\begin{itemize}
		\item rappresenta un'ottima opportunità per apprendere linguaggi e strumenti nuovi;
		\item utilizza strumenti attuali di machine learning e intelligenza artificiale;
		\item è un'applicazione innovativa e unica nel suo genere, non ne esistono di simili e ha grandi potenzialità di utilizzo;
		\item l'azienda proponente è disponibile e offre supporto nello sviluppo e minicorsi per lo sviluppo back-end;
	\end{itemize}
\section{Impegni}

	\subsection{Ore produttive}
	Ogni membro del gruppo si impegna a lavorare per 100 ore totali a membro, suddivise nei ruoli richiesti dal progetto nel seguente modo: 
		\begin{center}
			\begin{tabular}{||c | c | c | c | c||} 
			 \hline
			 Ruolo & Costo orario  & Ore per ruolo & Ore per membro & Costo totale \\ [0.5ex] 
			 \hline\hline
			 Responsabile & 30 & 42 & 6 & 1260 \\ 
			 \hline
			 Amministratore & 20 & 42 & 6 & 840 \\
			 \hline
			 Analista & 25 & 112 & 16 & 2800 \\
			 \hline
			 Progettista & 25 & 133 & 19 & 3325 \\
			 \hline
			 Programmatore & 15 & 196 & 28 & 2940 \\
			 \hline
			 Verificatore & 15 & 175 & 25 & 2625 \\
			 \hline\hline
			 Totale &  & 700 & 100 & 13790 \\
			 \hline
			\end{tabular}
		\end{center}

\subsection{Preventivo costi e scadenza}
Il gruppo stima un preventivo di 13790 euro per 700 ore di lavoro totali e si impegna a consegnare il prodotto software entro il 30 Settembre 2022.



\end{document}
